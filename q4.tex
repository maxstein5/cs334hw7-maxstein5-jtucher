\documentclass[10pt]{article}

% Lines beginning with the percent sign are comments
% This file has been commented to help you understand more about LaTeX

% DO NOT EDIT THE LINES BETWEEN THE TWO LONG HORIZONTAL LINES

%---------------------------------------------------------------------------------------------------------

% Packages add extra functionality.
\usepackage{times,graphicx,epstopdf,fancyhdr,amsfonts,amsthm,amsmath,algorithm,algorithmic,xspace,hyperref}
\usepackage[left=1in,top=1in,right=1in,bottom=1in]{geometry}
\usepackage{sectsty}	%For centering section headings
\usepackage{enumerate}	%Allows more labeling options for enumerate environments 
\usepackage{epsfig}
\usepackage[space]{grffile}
\usepackage{booktabs}
\usepackage{qtree}

% This will set LaTeX to look for figures in the same directory as the .tex file
\graphicspath{.} % The dot means current directory.

\pagestyle{fancy}

\lhead{\YOURID}
\chead{Question \RRNumber}
\rhead{\today}
\lfoot{CSCI 334: Principles of Programming Languages}
\cfoot{\thepage}
\rfoot{Fall 2018}

% Some commands for changing header and footer format
\renewcommand{\headrulewidth}{0.4pt}
\renewcommand{\headwidth}{\textwidth}
\renewcommand{\footrulewidth}{0.4pt}

% These let you use common environments
\newtheorem{claim}{Claim}
\newtheorem{definition}{Definition}
\newtheorem{theorem}{Theorem}
\newtheorem{lemma}{Lemma}
\newtheorem{observation}{Observation}
\newtheorem{question}{Question}

\setlength{\parindent}{0cm}


%---------------------------------------------------------------------------------------------------------

% DON'T CHANGE ANYTHING ABOVE HERE

% Edit below as instructed
\newcommand{\YOURID}{} 	% Replace 1234567 with your ID for the course
\newcommand{\RRNumber}{4}	% Replace 0 with the actual problem set #
\newcommand{\ProblemHeader}	% Don't change this!

\begin{document}
	
\vspace{\baselineskip}	% Add some vertical space

\begin{enumerate}
\item[(a)] Under static scoping, the value of result() is 16. On line 3 in the function declaration \verb|let f = fun y -> x + y|, x takes the value 2 because that is how it was defined in the program text a line earlier, so whenever that function is called it will substitute 2 in for x. On line 5, x takes on the value 7 because it was defined as 7 in the previous line, making this the closest enclosing scope of the text. Same with line 6, where x takes on the value 7 based on the definition in line 4. Therefore the line \verb|x + f x| will evaluate to \verb|7 + (fun y -> 2 + y) 7 = 7 + 9 = 16|.

\item[(a)] Under dynamic scoping, x + f x would be 21, because the most recent value for x on the stack is 7, so the line \verb|x + f x = 7 + (fun y -> 7 + y) 7 = 7 + 7 + 7 = 21|. For line 3, x is 2, so the function f is defined \verb|let f = fun y -> 2 + y|, but then in line 5 and 6, the most recent definition on the stack has x as 7, so line 5 is \verb|7 +| and line 6 is \verb|x+ (fun y -> x + y) x| and 7 is substituted in for all x’s.

\item[(a)] F\# uses static scope. One way I know is if we think about curried functions, and the type \verb|'a -> ( 'b -> 'c) |. We can imagine a curried function of this type, \verb|f x y|, and then imagine only passing it a value for x, meaning that it will return a function \verb| ( 'b -> 'c)|. We can then use this new function, and it will not be affected by associating x with a new stack frame. This implies that F\# is using the closest value in the program text, not the most recent on the stack.

\end{enumerate}

% DO NOT DELETE ANYTHING BELOW THIS LINE
\end{document}
